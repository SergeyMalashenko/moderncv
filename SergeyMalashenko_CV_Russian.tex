%% start of file `template.tex'.
%% Copyright 2006-2015 Xavier Danaux (xdanaux@gmail.com), 2020-2021 moderncv maintainers (github.com/moderncv).
%
% This work may be distributed and/or modified under the
% conditions of the LaTeX Project Public License version 1.3c,
% available at http://www.latex-project.org/lppl/.


\documentclass[11pt,a4paper,sans]{moderncv}        % possible options include font size ('10pt', '11pt' and '12pt'), paper size ('a4paper', 'letterpaper', 'a5paper', 'legalpaper', 'executivepaper' and 'landscape') and font family ('sans' and 'roman')

% moderncv themes
\moderncvstyle{classic}                             % style options are 'casual' (default), 'classic', 'banking', 'oldstyle' and 'fancy'
\moderncvcolor{blue}                               % color options 'black', 'blue' (default), 'burgundy', 'green', 'grey', 'orange', 'purple' and 'red'
%\renewcommand{\familydefault}{\sfdefault}         % to set the default font; use '\sfdefault' for the default sans serif font, '\rmdefault' for the default roman one, or any tex font name
%\nopagenumbers{}                                  % uncomment to suppress automatic page numbering for CVs longer than one page

% character encoding
%\usepackage[utf8]{inputenc}                       % if you are not using xelatex ou lualatex, replace by the encoding you are using
%\usepackage{CJKutf8}                              % if you need to use CJK to typeset your resume in Chinese, Japanese or Korean

% adjust the page margins
\usepackage[scale=0.75]{geometry}
\setlength{\footskip}{136.00005pt}                 % depending on the amount of information in the footer, you need to change this value. comment this line out and set it to the size given in the warning
\setlength{\hintscolumnwidth}{4cm}                % if you want to change the width of the column with the dates
%\setlength{\makecvheadnamewidth}{10cm}            % for the 'classic' style, if you want to force the width allocated to your name and avoid line breaks. be careful though, the length is normally calculated to avoid any overlap with your personal info; use this at your own typographical risks...

% font loading
% for luatex and xetex, do not use inputenc and fontenc
% see https://tex.stackexchange.com/a/496643
\ifxetexorluatex
  \usepackage{fontspec}
  \usepackage{unicode-math}
  \defaultfontfeatures{Ligatures=TeX}
  \setmainfont{Latin Modern Roman}
  \setsansfont{Latin Modern Sans}
  \setmonofont{Latin Modern Mono}
  \setmathfont{Latin Modern Math} 
\else
  \usepackage[utf8]{inputenc}
  \usepackage[T1]{fontenc}
  \usepackage{lmodern}
\fi
\usepackage[russian]{babel}

% personal data
\name{Сергей}{Малашенко}
\title{Резюме}                               % optional, remove / comment the line if not wanted
\born{20 июля 1984}                                 % optional, remove / comment the line if not wanted
\address{ул.Силкина 8а, кв. 67}{607190, Саров, Нижегородская область}{Россия}% optional, remove / comment the line if not wanted; the "postcode city" and "country" arguments can be omitted or provided empty
\phone[mobile]{+7~(909)~294~79~72}                   % optional, remove / comment the line if not wanted; the optional "type" of the phone can be "mobile" (default), "fixed" or "fax"
\email{malashenko\_sergei@yahoo.com}                               % optional, remove / comment the line if not wanted

% Social icons
\social[linkedin]{sergey-malashenko}                        % optional, remove / comment the line if not wanted
\social[telegram]{sergey\_malashenko}                            % optional, remove / comment the line if not wanted
\photo[70pt][0.4pt]{pictures/DKC_0122.JPG}                       % optional, remove / comment the line if not wanted; '64pt' is the height the picture must be resized to, 0.4pt is the thickness of the frame around it (put it to 0pt for no frame) and 'picture' is the name of the picture file
%\quote{Some quote}                                 % optional, remove / comment the line if not wanted

% bibliography adjustments (only useful if you make citations in your resume, or print a list of publications using BibTeX)
%   to show numerical labels in the bibliography (default is to show no labels)
%\makeatletter\renewcommand*{\bibliographyitemlabel}{\@biblabel{\arabic{enumiv}}}\makeatother
\renewcommand*{\bibliographyitemlabel}{[\arabic{enumiv}]}
%   to redefine the bibliography heading string ("Publications")
%\renewcommand{\refname}{Articles}

% bibliography with mutiple entries
%\usepackage{multibib}
%\newcites{book,misc}{{Books},{Others}}
%----------------------------------------------------------------------------------
%            content
%----------------------------------------------------------------------------------
\begin{document}
%\begin{CJK*}{UTF8}{gbsn}                          % to typeset your resume in Chinese using CJK
%-----       resume       ---------------------------------------------------------
\makecvtitle

\section{Образование}
\cventry{2002--2007}{математик, системный программист}{Саровский Физико-Технический Институт (МИФИ)}{}{\textit{GPA -- 4.95}}{}  % Arguments not required can be left empty
\cventry{2011--2015}{аспирантура}{Нижегородский государственный университет им. Н. И. Лобачевского}{}{}{}
\cventry{2020--Present}{data science and data engineering}{OZON Masters}{}{}{
\begin{itemize}
\item Машинное обучение, Глубокое обучение
\item Математическая статистика
\item Вычислительная линейная алгебра
\item Big Data and Data Engineering
\end{itemize}
}

\section{Опыт}
\subsection{Профессиональный}
\cventry{2018--Present}{Старший специалист отдела видеоаналитики}{\textsc{Erlyvideo}}{\url{https://flussonic.ru/}}{}{
Под моим руководствои и непосредственном участии разработана система детектирования и распознавания регистрационных знаков (многонациональная). Собраны и обработаны необходимые наборы данных, разработан инструмент для генерации искусственных изображений регистрационных знаков. Проанализированы доступные научные публикации и построенны требуемые модели для детектирования объектов и распознавания текста.
Также была разработана система детектирования и раcпознавания лиц. Обе системы доведены до конечного продукта.}

\cventry{2015--2018}{Senior Software Engineer}{\textsc{V5Systems}}{\url{https://v5systems.us/}}{}{
Под моим руководством и непосредственном участии разработана система для детектирования объектов (человек, машина) на встраиваемых системах (Nvidia Jetson TX1, TX2). Собраны и обработанны необходимые наборы данных, построенны компактные модели, разработана собственная библиотека, реализущая прямой вывод модели (model inference), также реализованны алгоритмы слежения за объектами (object tracking). Видеоаналитика использовалась в автономных системам, которые питались от солнечных батарей}

\cventry{2011--2015}{Senior Software Engineer}{\textsc{ЗАО Интел}}{\url{https://intel.com/}}{}{
Участвовал в разработке численного решателя системы уравнений, описывающих процесс электромиграции. Проанализировал систему уравнений, предложил алгоритм решения и реализовал его.
\newline{}
Был разработчиком библиотеки Level Set методов, которая позволяет решать задачу эволюции поверхностей произвольной формы. Эволюция описывается дифференциальным уравнением гиперболического типа, в рамках библиотеки были реализованны разностные схемы ENO и WENO типа. \newline{}
Участвовал в разработке библиотеки MOST, реализующей геометрические примитивы и алгоритмы над ними, библиотека использовалась для построения расчетной сетки. В ряд ключевых алгоритмов добавил поддержку точной арифметики (exact real arithmetic).
}

\cventry{2007-- 2011}{Младший научный сотрудник}{\textsc{РФЯЦ-ВНИИЭФ}}{\url{http://www.vniief.ru/}}{}{
Участвовал в разработке численного решателя системы уравнений газовой динамики и теплопроводности. Выполнил распараллеливание вычислительного ядра и сервисных алгоритмов, используя технологии OpenMP и MPI
\newline{}
Был руководителем совместного проекта с ОКБМ им. Африкантова. В рамках проекта применил теорию подобия для анализа поставленной задачи, нашим коллективом были выполнены численные эксперименты, что позволило обосновать применимость некоторых моделей турбулентности для решения поставленной задачи.
}
\cventry{2007--2008}{инженер-программист}{ЗАО «ИНКОМЕТ»}{}{}{
Реализовал программный пакет, в рамках программно-аппаратного комплекса ИП-4, для оценивания термического расширения материала. Выполнил численные эксперименты на оборудовании ПАО «Новолипецкий металлургический комбинат»}

\section{Навыки}
\cvitem{Математические навыки}{машинное обучение, глубокое обучение, нейронные сети, метод конечных элементов, метод конечных объемов,  системы дифференциальных уравнений в частных производных, Level Set методы}
\cvitem{Языки программирования}{\textsc{C/C++}, \textsc{Python}, \textsc{Bash}, \textsc{Lua}, \LaTeX}

\section{Языки}
\cvitemwithcomment{Русский}{родной}{}
\cvitemwithcomment{Английский}{средний}{читаю профессиональную литературу, могу проходить интервью}
\end{document}


%% end of file `template.tex'.

